\section{Problème de type {\em flow-shop} à deux et trois machines}
\subsection{Problème}
Les problèmes de type {\em flow-shop} comportent un ensemble de tâches se
séparant en étapes, toutes les tâches ayant le même nombre d'étapes. Chaque
étape doit être traitée sur une machine différente et dans le même ordre pour
toutes les tâches. Il y a donc autant de machines que d'étapes. Chaque étape ne
commence que lorsque la précédente est terminée et lorsque la machine nécessaire
est libre, l'objectif est de minimiser les moments où une machine ne fait rien
afin d'optimiser l'occupation totale des machines et de réduire le temps total
d'exécution.

L'algorithme de Johnson que nous allons présenter est une solution optimale pour
le cas à deux machines et fournit une heuristique décente et rapide à calculer
pour le cas à trois machines.

On note enfin que les tâches ne sont pas préemptibles et que le fait que l'ordre
des tâches soit identique sur les deux ou trois machines est une restriction de
l'algorithme de Johnson mais pas du problème posé, il se peut donc que pour
trois machine la solution optimale implique un ordre différent selon la
machine.

\subsection{Présentation}
L'algorithme de Johnson consiste à créer deux listes $T$ et $R$ et de répartir
chacune des tâches selon qu'elles soient plus courtes sur la première ou la
seconde machine. On trie ensuite $T$ par ordre croissant selon $p_{i1}$ et $R$ par
ordre décroissant selon $p_{i2}$. Et enfin on retourne leur concaténation.
Ce résultat nous donne $L$ l'ordre des tâches à effectuer sur {\em m1},
dès qu'une tâche a été traitée sur {\em m1} elle passe sur {\em m2}.

\subsection{Algorithme}
\begin{algorithm}
\caption{Algorithme de Johnson}
\begin{algorithmic}
\STATE $X \leftarrow \{1,...,n\}$
\STATE $T \leftarrow O$
\STATE $R \leftarrow O$
\WHILE {$X \neq O$}
	\STATE Choisir $i^*$, $j^*$ avec $p_{i^*,j^*} = min\{p_{ij}|i \in X; j = 1,2\}$
	\IF {$j^* = 1$}
		\STATE $T \leftarrow T + i^*$
	\ELSE
		\STATE $R \leftarrow R + i^*$
	\ENDIF
	\STATE $X \leftarrow X\backslash\{i^*\}$
\ENDWHILE
\STATE $Trier(T, (p_{i1} - p_{j1}))$
\STATE $Trier(R, (p_{j2} - p_{i2}))$
\STATE retourner $T + R$
\end{algorithmic}
\end{algorithm}

\subsection{Exemple}
Appliquons maintenant cet algorithme sur l'instance :

\begin{center}
\begin{tabular}{|c|c|c|}
\hline
$i$ & $p_{i1}$ & $p_{i2}$ \\
\hline
1 & 4 & \textbf{3} \\
\hline
2 & \textbf{3} & 3 \\
\hline
3 & \textbf{4} & 4 \\
\hline
4 & \textbf{1} & 4 \\
\hline
5 & \textbf{8} & 8 \\
\hline
\end{tabular}
\end{center}

On commence par remplir $T$ avec tous les $i$ tel que $p_{i1} \le p_{i2}$, et
$R$ par tous les autres.
On obtient $T = \{2,3,4,5\}$ et $R = \{1\}$.
On les trie et fusionne, ce qui donne la liste des tâches 
suivante $L = \{4,2,3,5\} + \{1\}$.

\subsection{Implémentation}
Notre implémention travaille sur un tableau de tâches (selon les spécifications de
tamandua), l'utilisation de listes est donc simulée.
Nous travaillons directement sur le résultat concatené, ce qui implique une
minimisation des données nécessaires. On opère un tri des tâches en respectant
l'algorithme de Johnson, c'est à dire que le début du tableau représente la liste T,
et R la fin. Pour ce faire l'algorithme de tri utilise le prédicat de
comparaison suivant :

\begin{algorithm}
\caption{Fonction de comparaison de Johnson-sort}
\begin{algorithmic}
\IF {$p_i\ est\ dans\ T$}
	\IF {$p_{i+1}\ est\ dans\ T\ et\ (p_{i,1} > p_{i+1,1})$}
		\STATE \COMMENT{ La tâche1 et la tâches2 sont toutes les deux dans T, \\
		Et la tâche1 est moins courte que la tâche2 (sur m1). }
		\STATE Placer $p_i$ après $p_{i+1}$
	\ELSE
		\STATE \COMMENT{ La tâche1 est dans T mais non la tâche2, \\
		donc la tâche1 est forcément avant ; \\
		Ou la tâche1 est plus courte que la tâche2. }
		\STATE Placer $p_i$ avant $p_{i+1}$
	\ENDIF
\ELSE
	\IF {$p_{i+1}\ est\ dans\ T\ ou\ (p_i{i,2} < p_{i+1,2})$}
		\STATE \COMMENT{ La tâche1 est dans R et non la tâche2, \\
		donc la tâche1 est forcement après ; \\
		Ou la tâche1 est plus courte que la tâche2 (sur m2). }
		\STATE Place $p_i$ après $p_{i+1}$
	\ELSE
		\STATE \COMMENT{ La tâche1 et la tâche2 sont toutes les deux dans R, \\
		 Et la tâche1 est moins courte que la tâche2. }
		\STATE Placer $p_i$ avant $p_{i+1}$ 
	\ENDIF
\ENDIF
\end{algorithmic}
\end{algorithm}

Une tâche répond au prédicat {\em est dans T} si elle s'execute plus rapidement
sur {\em m1} que sur {\em m2}.
Nous avons une compléxité en $O(1)$ pour le calcul de la comparaison, tout se joue
dans le tri. Un tri ayant optimalement une complexité de $O(n*log(n))$, c'est
aussi la complexité de l'algorithme de Johnson.

\subsection{Extention à trois machines}
Pour étendre l'algorithme à plus de deux machines, on compare la somme du temps
sur les deux premières machines avec la somme du temps sur les deux secondes
machines.
