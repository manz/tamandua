\section{Ordonnancement selon l'algorithme de Johnson}
\subsection{Problème}
L'algorithme de Johnson s'utilise pour les problèmes de type {\em flow-shop}.
On doit ordonnancer $n$ tâches de manière à ce qu'elles soient traitées par
toutes les machines $m$.
Une tâche ne peut être executée que sur une seule machine à la
fois. En un temps optimal. Encore une fois, on n'a pas d'interdépendance entre
les tâches. Cet algorithme est une méthode heuristique qui s'applique 
dans le cas où $m=2$.

\subsection{Présentation}
L'algorithme de Johnson consiste à créer deux listes $T$ et $R$. Et d'y placer 
le numéro des tâches en fonction de leurs durée minimale entre les deux 
machines. On tri ensuite $T$ par ordre croissant selon $p_{i1}$ et $R$ par
ordre decroissant selon $p_{i2}$. Et enfin on retourne leurs concaténation.
Ce résultat nous donne $L$ l'ordre des tâches à effectuer sur {\em m1},
dès qu'une tâche a été traité sur {\em m1} elle passe sur {\em m2}.

\subsection{Algorithme}
\begin{algorithm}
\caption{Algorithme de Johnson}
\begin{algorithmic}
\STATE $X \leftarrow \{1,...,n\}$
\STATE $T \leftarrow \O$
\STATE $R \leftarrow \O$
\WHILE {$X \neq \O$}
	\STATE Choisir $i^*$, $j^*$ avec $p_{i^*,j^*} = min\{p_{ij}|i \in X; j = 1,2\}$
	\IF {$j^* = 1$}
		\STATE $T \leftarrow T + i^*$
	\ELSE
		\STATE $R \leftarrow R + i^*$
	\ENDIF
	\STATE $X \leftarrow X\backslash\{i^*\}$
\ENDWHILE
\STATE TRIER(T, ($p_{i1} - p_{j1}$))
\STATE TRIER(R, ($p_{j2} - p_{i2}$))
\STATE retourner $T + R$
\end{algorithmic}
\end{algorithm}

\subsection{Implémentation}
Notre implémention travail sur un tableau de tâches (selon les spécifications de
tamandua), l'utilisation de listes est donc simulée.
Nous travaillons directement sur le résultat concatené, ce qui implique une
minimisation des données nécessaires. On opère un tri des tâches en respectant
l'algorithme de Johnson, soit que le début du tableau représente la liste T,
et R la fin. Pour ce faire l'algortithme de tri doit se baser sur cet algorithme
de comparaison :

\begin{algorithm}
\caption{Fonction de comparaison de Johnson-sort}
\begin{algorithmic}
\IF {$p_i\ est\ dans\ T$}
	\IF {$p_{i+1}\ est\ dans\ T\ et\ (p_{i,1} > p_{i+1,1})$}
		\STATE // La tâche1 et la tâches2 sont toutes les deux dans T,
		\STATE // Et la tâche1 est moins rapide que la tâche2 (sur m1).
		\STATE Placer $p_i$ après $p_{i+1}$
	\ELSE
		\STATE // La tâche1 est dans T mais non la tâche2,
		\STATE // donc la tâche1 est forcément avant ;
		\STATE // Ou la tâche1 est plus rapide que la tâche2.
		\STATE Placer $p_i$ avant $p_{i+1}$
	\ENDIF
\ELSE
	\IF {$p_{i+1}\ est\ dans\ T\ ou\ (p_i{i,2} < p_{i+1,2})$}
		\STATE // La tâche1 est dans R et non la tâche2,
		\STATE // donc la tâche1 est forcement après ;
		\STATE // Ou la tâche1 est plus rapide que la tâche2 (sur m2).
		\STATE Place $p_i$ après $p_{i+1}$
	\ELSE
		\STATE // La tâche1 et la tâche2 sont toutes les deux dans R,
		\STATE // Et la tâche1 est moins rapdiqe que la tâche2.
		\STATE Placer $p_i$ avant $p_{i+1}$
	\ENDIF
\ENDIF
\end{algorithmic}
\end{algorithm}

Une tâche répond au prédicat {\em est dans T} si elle s'execute plus rapidement
sur {\em m1} que sur {\em m2}.
Nous avons une compléxité en $O(1)$ pour cet algorithme, tout ce joue dans le
tri. Avec un {\em qsort} associé à un {\em mergesort} nous redonne la compléxité
de Johnon, soit $O(n.log_2(n))$.


\subsection{Exemple}
Appliquons maintenant cet algorithme sur l'instance :

\begin{center}
\begin{tabular}{|c|c|c|}
\hline
$i$ & $p_{i1}$ & $p_{i2}$ \\
\hline
1 & 4 & \textbf{3} \\
\hline
2 & \textbf{3} & 3 \\
\hline
3 & \textbf{4} & 4 \\
\hline
4 & \textbf{1} & 4 \\
\hline
5 & \textbf{8} & 8 \\
\hline
\end{tabular}
\end{center}

On commence par remplir $T$ avec tous les $i$ tel que $p_{i1} \le p_{i2}$, et
$R$ par tous les autres.
On obtient $T = \{2,3,4,5\}$ et $R = \{1\}$.
On les trie et fusionne, ce qui donne la liste des tâches 
suivante $L = \{4,2,3,5\} + \{1\}$.
