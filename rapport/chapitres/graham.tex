\section{Répartition de tâches sans précédences sur plusieurs machines}
\subsection{Problème}
Dans ce problème, il s'agit de répartir un ensemble de tâches de durées 
diverses, sans dépendances entre elles sur plusieurs machines identiques, 
l'objectif de l'ordonnancement étant de minimiser le temps total de calcul.
Nous allons tout d'abord utiliser une répartition de type premier arrivé-premier 
servi sur l'ensemble des tâches, puis utiliser l'algorithme de coffman-graham en 
utilisant la durée des taches comme relation d'ordre de priorité avant de les 
répartir.

\subsection{Algorithme naïf}
\subsubsection{Présentation}
Le principe est de prendre une liste de tâches dans un ordre quelconque et 
à chaque itération de prendre la première tâche de la liste et de l'affecter 
à la première machine disponible.
\subsubsection{Exemple}
Prenons ${n=13}$ ${m=4}$ et ${(p_{1},p_{2},...,p_{12},p_{13}) 
= (1,1,...,1,4)}$.\\
L'application de l'algorithme à cet ensemble donne le résultat suivant, ayant 
pour temps d'exécution total $7$ :
\begin{center}
$m_{1} = {p_{1},p_{5},p_{9},p_{13}}$ \\
$m_{2} = {p_{2},p_{6},p_{10}}$ \\
$m_{3} = {p_{3},p_{7},p_{11}}$ \\
$m_{4} = {p_{4},p_{8},p_{12}}$ \\
\end{center}
Or on peut facilement produire à la main un meilleur résultat :
\begin{center}
$m_{1} = {p_{1},p_{4},p_{7},p_{10}}$ \\
$m_{2} = {p_{2},p_{5},p_{8},p_{11}}$ \\
$m_{3} = {p_{3},p_{6},p_{9},p_{12}}$ \\
$m_{4} = {p_{13}}$ \\
\end{center}
Celui-ci a un temps d'exécution total de $4$, les bornes inférieures que nous 
allons donner par la suite pour $C^*_{max}$ montrent que c'est un résultat 
optimal.
\subsubsection{Mesure du ratio de performance}
Commençons tout d'abord par poser deux bornes inférieures pour $C^*_{max}$, 
toutes deux relativement triviales. La première est déterminée par le cas 
optimal où toutes les machines sont occupées pendant toute la durée du calcul, 
et donc où toutes les machines s'arrètent de travailler en même temps.
\begin{equation}
\label{graham-chien}
C^*_{max} \geq \frac{\sum_{i=1}^n p_{i}}{m}
\end{equation}
Une seconde borne encore plus triviale est la durée de la plus longue de toutes 
les tâches.
\begin{equation}
\label{graham-lapin}
C^*_{max} \geq max^n_{i=1}(p_{i})
\end{equation}
Soit $j'$ la tâche qui se termine à $C^{LS}_{max}$ :
\begin{equation}
C^{LS}_{max} = t_{j'} + p_{j'}
\end{equation}
Nous avons de plus :
\begin{equation}
p_{j'} \leq max^n_{i=1}(p_{i})
\end{equation}
Ce qui combiné avec \eqref{graham-lapin} nous donne le résultat suivant :
\begin{align}
p_{j'} &\leq C^*_{max} \\
\label{graham-canard}
C^{LS}_{max} &\leq t_{j'} + C^*_{max}
\end{align}
Or $t_{j'}$ est le moment ou $j'$ débute, ce qui signifie que lors de son 
affectation à une machine toute les autres machines sont déja occupées jusqu'à 
au moins $t_{j'}$ ce qui signifie qu'à cet instant les machines ont déja calculé 
pendant une durée de $m*t_{j'}$, or le temps de travail des machines est 
forcément inférieur à la durée total de calcul des tâches, ce qui nous donne :
\begin{align}
m*t_{j'} &\leq \sum_{i=1}^n p_{i} \\
\label{graham-ours}
t_{j'} &\leq \frac{\sum_{i=1}^n p_{i}}{m} \\
t_{j'} &\leq C^*_{max}
\end{align}
Ce qui combiné avec \eqref{graham-canard} nous donne enfin le ratio de 
performance :
\begin{align}
C^{LS}_{max} &\leq 2*C^*_{max} \\
\frac{C^{LS}_{max}}{C^*_{max}} &\leq 2 
\end{align}
\subsubsection{Complexité de l'implémentation}
Pour implémenter cet algorithme, nous avons besoin de deux structures de 
données, la première est la liste des tâches, étant donné que nous ne voulons 
que prendre le premier élément, une pile ou une itération sur un tableau sont 
suffisantes. La seconde structure est celle qui permet de savoir à chaque 
itération sur un élément de la liste des tâches quelle est la première machine 
disponible, nous avons choisit pour cela un tas binaire ordonné avec la première 
machine disponible en haut du tas.

Le tas binaire permet d'accéder à cet élément en temps constant, de le supprimer 
en $O(log(m))$ et de le réinserer une fois la durée de la nouvelle tâche ajoutée 
en $O(log(m))$.
L'opération complète que l'on nomme $Augmenter\_minimum(tas, duree)$ prends donc 
$O(log(m))$ et est répétée pour chaque tâche, ce qui nous fait une complexité 
totale de $O(n.log(m))$.

On note enfin que la création du tas initial avec toute les durées à 0 et avec 
un élément pour chaque machine se fait en $O(m)$ avec une implémentation en 
tableau du tas, $m$ étant typiquement inférieur à $n$ (sans quoi 
l'ordonnancement est trivial) cela n'influence pas la complexité.
\begin{algorithm}
\caption{Augmenter\_minimum(tas, duree)}
\begin{algorithmic}
\STATE $elem \leftarrow extraire\_minimum(tas)$
\STATE $machine \leftarrow machine(elem)$
\STATE $duree(elem) \leftarrow duree(elem) + duree$
\STATE $inserer(tas, elem)$
\STATE \textbf{retourner} $machine$
\end{algorithmic}
\end{algorithm}

\begin{algorithm}
\caption{Ordonnancement\_LS(taches)}
\begin{algorithmic}
\STATE $tas = creer\_tas()$
\FORALL{$tache$ \textbf{dans} $taches$}
\STATE $machine(tache) = Augmenter\_minimum(tas, duree(tache))$
\ENDFOR
\end{algorithmic}
\end{algorithm}
\subsection{Algorithme de Coffman-Graham}

\subsubsection{Présentation}
L'algorithme de Coffman-Graham est similaire au précédent, avec pour seul 
différence un tri préalable des tâches selon le durée, et ceci de manière 
décroissante. Le tri induit un calcul préalable de complexité $O(n.log(n))$ mais 
améliore, comme nous allons le démontrer le ratio de performance en l'abaissant 
à $\frac{4}{3}$.
\begin{algorithm}
\caption{Ordonnancement\_LPT(taches)}
\begin{algorithmic}
\STATE $Trier(taches, duree(b)-duree(a))$
\STATE $Ordonnancement\_LS(taches)$
\end{algorithmic}
\end{algorithm}
\subsubsection{Démonstration}
Nous pouvons tout d'abord partir de l'équation \eqref{graham-ours} trouvée lors 
de l'étude de $C^{LS}_{max}$ :
\begin{align}
C^{LPT}_{max} &= t_{j'} + p_{j'} \\
C^{LPT}_{max} &\leq \frac{\sum_{i=1}^n p_{i}}{m} + p_{j'} \\
\end{align}
Puis en employant \eqref{graham-chien}, nous obtenons :
\begin{equation}
C^{LPT}_{max} \leq C^*_{max} + p_{j'}
\end{equation}
Distinguons maintenant deux cas, selon la valeur de $p_{j'}$, tout d'abord si 
$p_{j'} \leq \frac{C^*_{max}}{3}$ :
\begin{align}
C^{LPT}_{max} &\leq C^*_{max} + \frac{C^*_{max}}{3} \\
C^{LPT}_{max} &\leq \frac{4}{3} C^*_{max} \\
\frac{C^{LPT}_{max}}{C^*_{max}} &\leq \frac{4}{3}
\end{align}
Cependant si $p_{j'} > \frac{C^*_{max}}{3}$ la solution n'est pas aussi simple.  
Tout d'abord on peut écarter le cas où $n\leq m$, dans ce cas on ne peut faire 
mieux que de mettre zéro ou une tâche sur chaque machine, la solution est 
triviale. Ètudions maintenant le cas contraire.

Appelons $M$ la machine traitant $j'$, celle-ci traite un certain nombre 
d'autres tâches. On sait que le temps d'exécution de $M$ est le plus élevé de 
toutes les machines et détermine $C^{LPT}_{max}$. On sait aussi que les autres 
machines ont toutes un temps d'exécution d'au moins $t_{j'}$. On sait enfin que  
toutes les tâches sur $M$ sont de durées supérieure à $p_{j'}$ étant donné que 
$j'$ est la dernière traitée et que les tâches sont traitées par durée 
décroissante.

On ne peut enlever une tâche de $M$ pour la mettre sur une autre machine, les 
autres machines ayant un temps d'exécution déja supérieur à $t_{j'}$ et les 
tâches de $M$ ayant toutes une durée supérieure à $p_{j'}$, le temps d'exécution 
de l'autre machine deviendrait supérieur à $t_{j'}+p_{j'}$ donc supérieur 
à $C^{LPT}_{max}$ ce qui ne raccourcirait pas l'ordonnancement.

Cela signifie donc que pour réduire la durée d'exécution de $M$, il faut 
permuter des tâches de $M$ avec celles d'autres machines. Si on permute une 
tâche $a$ de $M$ avec une tâche $b$ d'une autre machine, la tâche $b$ se doit 
d'être de durée supérieur à $p_{j'}$ sinon cela signifie qu'elle est démarre 
après $t_{j'}$ et donc que la machine $M'$ d'ou vient $b$ aura un temps 
d'exécution supérieur à $t_{j'}+p_{j'}$ ce qui ne raccourcirait toujours pas 
l'ordonnancement.

Enfin on remarque que s'il y a trois tâches ou plus avant $j'$ sur $M$, alors 
$t_{j'} > 3*p_{j'}$. Or $C^*_{max} \geq \frac{\sum_{i=1}^n p_{i}}{m} \geq 
t_{j'}$, ce qui nous amène enfin à $C^*_{max} \geq 3*p_{j'}$ ce qui contredit 
notre hypothèse sur $p_{j'}$.  Il y a donc au plus deux tâches avant $j'$ sur 
$M$.

Cela signifie donc que l'on peut avoir au plus deux permutations pour réduire la 
durée d'exécution de $M$ et qu'on l'on ne peut répartir l'excédent que sur au 
plus deux autres machines. Ètant donné que toutes les autres machines ont un 
temps d'exécution supérieur à $t_{j'}$ on ne peut pas réduire la durée de $M$ de 
plus de $\frac{2}{3}p_{j'}$. Ce qui nous amène à la conclusion :

\begin{align}
C^*_{max} &\geq C^{LPT}_{max} - \frac{2*p_{j'}}{3} \\
C^*_{max} &\geq C^{LPT}_{max} - \frac{2*C^*_{max}}{9} \\
\frac{11}{9}C^*_{max} &\geq C^{LPT}_{max} \\
\frac{12}{9}C^*_{max} &\geq C^{LPT}_{max} \\
\frac{4}{3} &\geq \frac{C^{LPT}_{max}}{C^*_{max}} \\
\end{align}

Nous avons ainsi démontré que le ration de performance de $C^{LPT}_{max}$ est 
inférieur à $\frac{4}{3}$ dans tout les cas, CQFD.
