\chapter{Tamandua}
\section{Architecture}
Afin de simplifier au maximum les implémentations de ces problèmes 
d'ordonnancements. Tamandua à été découpé en trois parties :
\begin{itemize}
   \item un noyau sous la forme d'une bibliothèque dynamique
   \item des greffons implémentant les problèmes
   \item des clients se chargeant de rendre le résultat
\end{itemize}

\subsection{Le noyau}
Le noyau de l'application comprend {\em td\_base} et {\em td\_core}. Cette
bibliothèque fournis un format et un ensemble d'outils, pour implémanter
facilement des problèmes d'ordonnancements. \\
Le principe est de créer un {\em générateur}, qui n'est rien qu'une structure de
données contenant les paramètres à utiliser pour générer un ensemble de tâches.
Ces paramètres comportent entre autre le nombre de machines, le nombre de
{\em populations} de tâches à générer. Et pour chaques populations le bornage 
des poids et des durées de ces tâches.\\
Avoir des populations distinctes permet de créer des cas d'ordonnancement
aléatoire plus complexe. Par exemple on peut prendre un cas avec 2 populations, 
l'une contenant beaucoup de courtes tâches et l'autre peu de très longues 
tâches. Bien entendu on peu aussi obtenir ce cas avec un seul bornage, mais pas 
de manière aussi sûr.\\
Une fois tous les paramètres réglés, le générateur va créer un {\em job}
contenant un jeu de tâches correspondant et généré aléatoirement. Ce job est
ensuite passé aux greffons des problèmes pour être traité.

\subsection{Les greffons}
Plus connu sous le nom de plugins. Ils implémantent une interface prenant un job
et le retourne ordonné. Chargés à l'intialisation de la bibliothèque, ils
ne contiennent que le code de résolution du problème. Toutefois ils peuvent
comprendre plusieurs stratégies, donc plusieurs manière de résoudre un même
problème. 

