\chapter{Introduction}
\section{Problématique de l'ordonnancement}
Le sujet ici présenté est l'étude de plusieurs problèmes relatifs à
l'ordonnancement de tâches, c'est à dire la répartition et la permutation de
l'ordre d'un ensemble de tâches sur une ou plusieurs machines avec diverses
contraintes à satisfaire. Certaines contraintes étant strictes (dépendances entre
certaines tâches, nombre de machines ...), d'autres étant plus flexibles et nous servant d'objectif à
optimiser (durée total d'exécution principalement).

Parmis les problèmes
présentés, certains sont solubles en temps polynomiaux, et dans ce cas une
solution optimale est proposée, et d'autres obligent par contre à employer des heuristiques polynomiales afin de conserver des temps de calcul raisonnables. Ces heuristiques ont cependantes toutes une borne supérieure d'efficacité au moins inférieure au double de la solution optimale.

\section{Objectifs du projet}
Notre objectif tout d'abord est d'étudier ces problèmes, de leur trouver une
solution, de démontrer son optimalité ou sa borne maximale d'optimalité et enfin
la complexité de son implémentation. La seconde étape est l'implémentation
pratique de ces algorithmes et enfin une interface permettant de les tester et
de visualiser leur performance.
