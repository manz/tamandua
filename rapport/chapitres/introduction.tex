\chapter{Introduction}
\section{Description}
L'objectif de ce projet est d'étudier diverses classes de problèmes 
d'ordonnancement, ainsi que des algorithmes optimaux ou heuristiques permetant
de les résoudre en un temps polynomial. Ils sont généralement très complexes. 
Il est donc interessant de pouvoir en avoir une représentation graphique, à la
fois simple et clair, de leurs résultats. Nous avons donc créé un programme
permettant cela pour illustrer notre travail.

\section{Réalisation}
Trois problèmes ont été étudiés :
\begin{itemize}
\item les listes de Graham
\item la rêgle de Smith
\item l'algorithme de Johnson
\end{itemize}
A cela vient donc s'ajouter Tamandua. Une bibliothèque utilisant un système de
greffons, spécialement conçu pour modéliser les solutions d'oronnancements.
