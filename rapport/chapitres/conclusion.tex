\chapter{Conclusion}
\section{Réalisation du projet}
Concernant la réalisation générale du projet, nous pouvons considérer notre
objectif atteint, le noyau fonctionne bien et n'a jamais posé de problèmes
dès sa première implémentation, les interfaces sont assez complètes et pratiques
à utiliser.

Grâce à l'emploi de moyens de communication efficaces tels que
listes de diffusions et dépôt de fichiers versionné, nous avons pu rapidement
développer nos idées, corriger nos erreurs et améliorer sans cesse le produit
fini. Cependant nous regrettons quelques lacunes du coté algorithmique où
nous n'avons pas été capables de trouver une démonstration satisfaisante pour
certains problèmes.
\section{Améliorations futures}
Un grand manque de notre infrastructure pour l'implémentation d'autres problèmes
se situe dans la flexibilité de définition des tâches, peut-être qu'une
définition de structure de données récursives et des métadonnées arbitraires
permettrait cette généralisation. En effet la dichotomie entre tâches et étapes
est assez artificielle. De plus nous ne supportons pas la préemptibilité des tâches alors que c'est un concept important dans d'autres problèmes d'ordonnancement. 

La possibilité d'ajout de ces fonctionnalités a été évaluée mais un peu tard dans
le projet, or ce sont des problèmes assez lourds à ajouter, l'un comme l'autre
impliquant la refonte complète des structures de données d'échange entre le
noyau, les problèmes et surtout les interfaces graphiques.
